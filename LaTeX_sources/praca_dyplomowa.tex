\documentclass[12pt, en, eng]{mgr}
 
\begin{document}
\engtitle{
	An application for tracking the flow of resources for Bitcoin cryptocurrency \\
}
\title{Program do śledzenia przepływu środków w sieci Bitcoin}
\date{2018}
\author{Marcin Pieczka}
\supervisor{Dr inż., Radosław Michalski, Katedre Inteligencji Obliczeniowej}
\field{Informatyka (INF)}
\maketitle

\section{Goals}
Every process of analyzing data starts with accessing the data, when analyzing Bitcoin blockchain, this first step might be the hardest one. Currently Bitcoin blockchain contains over 160GB of raw, binary data and everyone who attempts to analyze it will have to have an efficient and reliable way to work with it. Moreover users should have obvious place for creating additional API's that will be placed on the same server as the data, so that operations that for instance require big amounts of data, but return results of significantly smaller size can be implemented efficiently, gathered in one place, and blend in with the system. 
\\
\\
My goal is to create application that: 
\begin{itemize}

\item
allows fast access to blockchain data
\item
allows access over the network
\item
updates its data in constant manner
\item
provides API for Python and R
\item
is easy to install

\end{itemize}

\section{Alternative software providing similar features}
\subsubsection*{blockchain.info}

Web application blockchain.info provides free access to Bitcoin blockchain data by either website, JSON API or API's dedicated to specific languages including Python, but without support for R. Number of requests is limited.
\\
\\
Relevant API's provided by blockchain.info:
\begin{itemize}
\item
getting single block, by block hash
\item
getting single block, by hight
\item
getting multiple block headers
\item
getting single transaction, by transaction hash
\item
getting all transactions of single or multiple addresses


\end{itemize}

Most common usage scenario in analytic context is getting range of blocks, for example all block from 10.04.2017 to 20.04.2017 nonetheless blockchain.info does not provide simple way to get such data. To achieve this we would have to make thousands of requests to API providing us with single block data, and this wouldn't be fast enough.

Other and the biggest problem is the API call limit that would make working with this application impossible for larger query's

\subsubsection*{blockexplorer.com}

This web application is very similar to blockchain.info in almost every aspect, although there are some differences. Data is accessible either by website or by JSON API, but there is no Python or R API provided. Set of API's is almost identical to blockchain.info, and does not provide easy way to get multiple blocks. The main difference is that there is no official API call limit, but using tool that is not controlled by us means that such limit can appear every moment.

\subsubsection*{libbitcoin-database}

This piece of software, after installation builds in-memory database of Bitcoin blockchain, its description promises high performance. Because of almost non existing documentation it is hard to write much about it.
\\
\\
Although this software is actively developed, and at first sight it might seam like solution that fits needs of our users, there are couple problems that make use of it problematic. One drawback is coming from its biggest selling point - it, being in-memory database makes it require large amounts of RAM, probably over 160GB, witch makes it not viable for hardware that is accessible for me. Next problem is it's API, with allows connecting to this db via C/C++ library, witch would make usage in R and Python at least problematic. Another problem is lack of useful documentation, witch would make usage a really hard experience.

\subsubsection*{BitcoinDatabaseGenerator}

BitcoinDatabaseGenerator is data transfer tool that can feed SQL database with blockchain data. It was written in C\# and as its author states it is only meant to be run on Windows machines. From the documentation we know that this software should be ran every time we want to update our database. Database schema that is created after operation consists of separate tables for blocks, transactions, transaction inputs and so forth, witch would require joining those tables in multitude of usage scenarios.
\\
\\
Although you can easily connect to SQL database from both R and Python, I find working with SQL databases in such case cumbersome, it would require abstracting away relational structure of data to more object like one. 


\section{Design}

\subsection{User requirements}
\subsubsection{User stories}
As a user, I want to access blockchain data from my Python/R script so that data analysis and data accessing can be made in the same code.
\\
\\
As a user, I want the data be in a format idiomatic to Python/R so that I don't have to convert it after.
\\
\\
As a user, I want easy access to block data from range of time or block height so that I can spend my time working with the data, and not with accessing it.
\\
\\
As a user, I want the installation not to require complicated operations so that I can do it without specialized knowledge.
\\
\\
As a user, I want to be able to run this software on my linux server so that I don't have to learn new operating system to use it.
\\
\\
As a user, I want fast access to the data so that my analytic scripts will be pleasurable to work with.
\\
\\
As a future maintainer, I want to have obvious place to create my own API's on server so that my new data hungry features can be placed on the same server as data for better performance.
\\
\\
As a future maintainer, I want some goal so that some reason.
\\
\\
As a user, I want some goal so that some reason.
\\
\\

\subsection{System inputs and outputs}

In this section I will specify inputs and outputs of the system, witch will then help me to discover what transformation the input data will undergo, and what components are needed to provide outputs efficiently

\subsubsection{Inputs}
\paragraph{Bitcoind BLK files}
The only source of bitcoin block data will be BLK files stored by full bitcoin node. Daemon process bitcoind gets blocks from neighboring nodes and stores them in data directory. The blk files in default configuration store up to 128MB of raw network format block data, and blocks are stored in order in witch they came from network. There exist multiple library's that handle parsing these files, so accessing this data will not be a problem.
\\
\\
One important thing to mention about bitcoin transactions stored in blocks is the fact that address from witch the transaction comes from is described as output of some previous transaction.

\paragraph{Request for blocks} will be the way user communicates with the system. In this request user will specify witch blocks he wants to receive, usually it will be blocks from specified range of time, or range of height.

\subsubsection{Outputs}
\paragraph{Response on request for blocks} will contain block data in format understandable by user 

\subsubsection*{Docker - for and against}

The main things that can be accomplished with well chosen deployment and packaging are simplicity of use and cross-platform possibilities witch can positively influence adoption rate of this solution in perspective of both usage and further open source development.

Docker is known for achieving those goals, but not everything about Docker will be helpful. My main concern is running Bitcoin Core in Docker Image, it would be a great inconvenience to people who already have full node on their machine and this problem has to be dealt with.

My proposal is to have Bitcoin Core installed separately, and create simple HTTP server that would provide access to block data, the rest of the application would by in Docker. Creation of such server can be achieved with running command  
\begin{verbatim}
python -m http.server 8000 --bind 127.0.0.1
\end{verbatim}
With that we can keep the advantages of using docker, and provide a new possibility to the user - keeping Bitcoin Core on separate machine witch when considering that both parts of the application, Bitcoin Core and database with API will be well over 100Gb in size might be a big advantage.

\section{Database}
\subsection*{Database operations}
My least concern are operations that modify the data, and their performance will not be taken into consideration

Main goal is to enable fast querying of the blocks by block hash, time and other block attributes, and to be able to return specified amount of consecutive blocks starting or ending with certain block. 

Additionally it might be necessary to enable fast querying for transactions or blocks containing transactions of certain addresses but further research of what users need in this regard is needed

\subsection{Database choice}
At the beginning lets simplify the choice between RDBMS and NoSQL databases. Out of many NoSQL possibilities I have chosen MongoDB database based on some quick research of different NoSQL systems strengths and weaknesses.

Lets lay out some facts that will help to decide whether to use relational database, or MongoDB

\begin{itemize}
\item 
To achieve fast querying, the data will be strongly denormalized
\item 
You can achieve comparable performance from MongoDB and some RDBMS but Mongo seams to make storing denormalized data idiomatic and RDBMS with highly denormalized data just don't feel right
\item MongoDB fully supports JSON, witch will be the format of data received by end user
\end{itemize}

Based on these facts I will use as my database MongoDB. This problem seams like a perfect usage for database of such type because of its denormalized nature and native support for JSON
 \subsection{Database structure}
 
 At this point I propose having one collection of blocks, each document containing all block attributes like hashes, time, transactions list and others
 
Indexing increases performance of querying the data and hinders the performance of operations like adding and removing data witch in this case looks like a great bargain. There might be additional memory cost associated with indexes, but this should not be a problem.

The indexes will be added to fields like block hash, time or height, adding indexing to transaction list is also a possibility and will be considered and tested. With indexes on transaction list it should be possible to quickly query for blocks containing transactions in witch given address receives or sends bitcoin, but its hard for me to speculate about this matter without thorough testing in live system.
\end{document}

